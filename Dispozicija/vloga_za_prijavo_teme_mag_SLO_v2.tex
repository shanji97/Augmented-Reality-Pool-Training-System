\documentclass[a4paper, 12pt]{article}
\usepackage[slovene]{babel}
\usepackage{lmodern}
\usepackage[T1]{fontenc}
\usepackage[utf8]{inputenc}
\usepackage{url}
\usepackage{xcolor}

\definecolor{munsell}{rgb}{0.0, 0.5, 0.69}
\newcommand\cmnt[1]{\textcolor{munsell}{#1}}


\topmargin=0cm
\topskip=0cm
\textheight=25cm
\headheight=0cm
\headsep=0cm
\oddsidemargin=0cm
\evensidemargin=0cm
\textwidth=16cm
\parindent=0cm
\parskip=12pt

\renewcommand{\baselinestretch}{1.2}

\begin{document}

%%%%%%%%%%%%%%%%%%%%%%%%%% Izpolni kandidat! %%%%%%%%%%%%%%%%%%%%%%%%%%
\newcommand{\ImeKandidata}{Aleksander} % Ime
\newcommand{\PriimekKandidata}{Kovač} % Priimek
\newcommand{\VpisnaStevilka}{63220478} % vpisna številka
\newcommand{\StudijskiProgram}{Multimedija, MAG} % Študijski program/smer
\newcommand{\NaslovBivalisca}{Lendavske gorice 409a, 9220 Lendava, Slovenija} % kaniddatov naslov
\newcommand{\SLONaslov}{Sistem za igranje namizne igre Osmica v obogateni resničnosti po ameriških pravilih} % naslov dela v slovenščini
\newcommand{\ENGNaslov}{Augmented Reality System for Playing Eight-Ball Pool with American Rules} % naslov dela v angleščini
%%%%%%%%%%%%%%%%%%%%%%%%%% Konec izpolnjevanja %%%%%%%%%%%%%%%%%%%%%%%%%%


\newcommand{\Kandidat}{\ImeKandidata~\PriimekKandidata}
\noindent
\Kandidat\\
\NaslovBivalisca \\
Študijski program: \StudijskiProgram \\
Vpisna številka: \VpisnaStevilka
\bigskip

{\bf Komisija za študijske zadeve}\\
Univerza v Ljubljani, Fakulteta za računalništvo in informatiko\\
Večna pot 113, 1000 Ljubljana\\

{\Large\bf
{\centering
    Vloga za prijavo teme magistrskega dela \\%[2mm]
\large Kandidat: \Kandidat \\[10mm]}}


\Kandidat, študent magistrskega programa na Fakulteti za računalništvo in informatiko, zaprošam Komisijo za študijske zadeve, da odobri predloženo temo magistrskega dela z naslovom:

%\hfill\begin{minipage}{\dimexpr\textwidth-2cm}
Slovenski: {\bf \SLONaslov}\\
Angleški: {\bf \ENGNaslov}
%\end{minipage}

Tema je bila že potrjena lani in je ponovno vložena: {\bf \textit{NE} }

Izjavljam, da je spodaj navedeni mentor predlog teme pregledal in odobril, ter da se z oddajo predloga strinja.

Magistrsko delo nameravam pisati v slovenščini. % In case you would like to write the thesis in English, comment this line out, and use the following template to explain your request:
%Komisijo zaprošam, da odobri pisanje magistrskega dela v angleškem jeziku z obrazložitvijo ... .

Za mentorja predlagam:

%%%%%%%%%%%%%%%%%%%%%%%%%% Izpolni kandidat! %%%%%%%%%%%%%%%%%%%%%%%%%%
\hfill\begin{minipage}{\dimexpr\textwidth-2cm}
Ime in priimek: doc. dr. Matevž Pesek\\
Ustanova: Fakulteta za računalništvo in informatiko, Univerza v Ljubljani\\
Elektronski naslov: matevz.pesek@fri.uni-lj.si
\end{minipage}
%%%%%%%%%%%%%%%%%%%%%%%%%% Konec izpolnjevanja %%%%%%%%%%%%%%%%%%%%%%%%%%

\bigskip


\hfill V Ljubljani, \today. \\
%V Ljubljani, dne …………………………
%
\hfill Podpis mentorja: \hspace{180px} \\ Podpis kandidata/kandidatke:    




\clearpage
\section*{PREDLOG TEME MAGISTRSKEGA DELA}

\section{Področje magistrskega dela}

slovensko: računalništvo in informatika, računalniška grafika, razširnjena resničnost, optično zaznavanje in sledenje predmetov, namizne igre\\
angleško: Computer Science and Informatics, Computer Graphics, Augmented Reality, Optical Object Detection and Tracking, Tabletop Games


\section{Ključne besede}

slovensko: obogatena resničnost,  \\
angleško:


\section{Opis teme magistrskega dela}

% Navodilo (pobrišite v končnem izdelku):
\cmnt{
\textbf{Briši iz končnega izdelka:}
Dolžina teme je zelo odvisna od zgoščenosti teksta in jasnosti podajanja argumentov. Zato je zelo težko predpisati natančno dolžino vsakega podpoglavja brez da bi ob tem posegali preveč v stil pisanja. Splošno vodilo naj bo, da naj bo iz teme: (i) jasno razviden problem in relevantnost problema, (ii) izpostavljene potencialne pomanjkljivosti sorodnih rešitev, (iii) novosti/prispevki naloge naj bodo jasni in v relaciji do sorodnih rešitev, (iv) jasne naj bodo pričakovane uporabljene metode za razvoj vaše rešitve, evalvacijo uspešnosti in primerjavo s sorodnimi deli.
\\
\\
Kljub temu v vsakem podpoglavju podajamo okvirni obseg v številu besed. Vodilo pa naj bo vseeno vsebinska kvaliteta.}
% V nadaljevanju opredelite izhodišča magistrskega dela in utemeljite znanstveno ali strokovno relevantnost predlagane teme.

\textbf{Pretekle potrditve predložene teme:}\\
Predložena tema ni bila oddana in potrjena v preteklih letih.
% Tu ne gre za to, da bi morali pogledati vse pretekle teme, če se slučajno ujemajo z vašo. Pač pa je ta del namenjen tistim, ki letos ponovno oddajate temo, ki vam jo je KŠZ potrdila že lani.
% v kolikor gre za temo, ki je bila že oddana v preteklem letu in je bila takrat potrjena, prosim to napišite. Prav tako napišite, če se v nečem tema razlikuje od lanske (ste kaj dodali, odvzeli).

\subsection{Uvod in opis problema}

%Navodilo:
\cmnt{Pojasnite, kaj je problem, ki ga želite reševati, in podajte motivacijo za delo. Pri opisu motivacije se navežite na literaturo in nerešene probleme, ki jih bo naslavljala vaša magistrska naloga. Delo umestite v ožje področje dela. Okvirni obseg: ~300 besed (1/2 strani A4).}

\subsection{Pregled sorodnih del}

%Navodilo:
\cmnt{Opišite pregled sorodnih del na ožjem področju, na katerem nameravate opravljati magistrsko nalogo. Vsako delo naj bo na kratko opisano v nekaj stavkih, besedilo pa naj poudari njegove glavne prednosti, slabosti ali posebnosti. Sklicujte se na dela, navedena v razdelku \ref{literatura} Literatura in viri. Pregled naj bo fokusiran.  Okvirni obseg: 1/2 - 2/3 strani A4.}

\subsection{Predvideni prispevki magistrske naloge}

%Navodilo:
\cmnt{Opišite predvidene prispevke magistrske naloge s področja računalništva in informatike, ki so lahko strokovni ali znanstveni. Poudarite in opišite predvideni napredek ali novost vašega dela v primerjavi z obstoječim stanjem na strokovnem (ali znanstvenem) področju.  Okvirni obseg: 70 besed.}


\subsection{Metodologija}

%Navodilo:
\cmnt{Na kratko opredelite metodologijo, ki jo nameravate uporabiti pri svojem delu. Metodologija vsebuje metode, ki jih nameravate uporabiti (npr. razvoj v izbranem programskem jeziku, izdelava strojne opreme itd.), postopek analize, postopek evalvacije vašega prispevka in primerjavo s sorodnimi deli.  Okvirni obseg: 1/4 - 1/3 A4 strani.}


\subsection{Literatura in viri}
\label{literatura}

%Navodilo:
\cmnt{Tu navedite vse vire, ki jih citirate v predlogu teme. Citiranje naj bo v skladu z znanstveno-strokovnimi standardi citiranja, na primer, \cite{Zivkovic2004}. Seznam naj vsebuje vsaj nekaj del, objavljenih v zadnjih petih letih. Prednostno naj bodo navedene objave s konferenc, revij, oziroma drugih priznanih virov.}

\renewcommand\refname{}
\vspace{-50px}
\bibliographystyle{elsarticle-num}
\bibliography{./bibliografija/bibliography}


%\bigskip
%
%Ljubljana, \today .

\end{document}
