\documentclass[a4paper, 12pt]{article}
\usepackage[slovene]{babel}
\usepackage{lmodern}
\usepackage[T1]{fontenc}
\usepackage[utf8]{inputenc}
\usepackage{url}
\usepackage{xcolor}

\definecolor{munsell}{rgb}{0.0, 0.5, 0.69}
\newcommand\cmnt[1]{\textcolor{munsell}{#1}}


\topmargin=0cm
\topskip=0cm
\textheight=25cm
\headheight=0cm
\headsep=0cm
\oddsidemargin=0cm
\evensidemargin=0cm
\textwidth=16cm
\parindent=0cm
\parskip=12pt

\renewcommand{\baselinestretch}{1.2}

\begin{document}

%%%%%%%%%%%%%%%%%%%%%%%%%% Izpolni kandidat! %%%%%%%%%%%%%%%%%%%%%%%%%%
\newcommand{\ImeKandidata}{Aleksander} % Ime
\newcommand{\PriimekKandidata}{Kovač} % Priimek
\newcommand{\VpisnaStevilka}{63220478} % vpisna številka
\newcommand{\StudijskiProgram}{Multimedija, MAG} % Študijski program/smer
\newcommand{\NaslovBivalisca}{Lendavske gorice 409a, 9220 Lendava, Slovenija} % kaniddatov naslov
\newcommand{\SLONaslov}{Sistem za igranje namizne igre Osmica v obogateni resničnosti po ameriških pravilih s pomočjo
očal Meta Quest 3 } % naslov dela v slovenščini
\newcommand{\ENGNaslov}{Augmented Reality System for Playing Eight-Ball Pool with American Rules
with the help of Meta Quest 3 glasses} % naslov dela v angleščini
%%%%%%%%%%%%%%%%%%%%%%%%%% Konec izpolnjevanja %%%%%%%%%%%%%%%%%%%%%%%%%%

\newcommand{\Kandidat}{\ImeKandidata~\PriimekKandidata}
\noindent
\Kandidat\\
\NaslovBivalisca \\
Študijski program: \StudijskiProgram \\
Vpisna številka: \VpisnaStevilka
\bigskip

{\bf Komisija za študijske zadeve}\\
Univerza v Ljubljani, Fakulteta za računalništvo in informatiko\\
Večna pot 113, 1000 Ljubljana\\

{\Large\bf
{\centering
    Vloga za prijavo teme magistrskega dela \\%[2mm]
\large Kandidat: \Kandidat \\[10mm]}}


\Kandidat, študent magistrskega programa na Fakulteti za računalništvo in informatiko, zaprošam Komisijo za študijske zadeve, da odobri predloženo temo magistrskega dela z naslovom:

%\hfill\begin{minipage}{\dimexpr\textwidth-2cm}
Slovenski: {\bf \SLONaslov}\\
Angleški: {\bf \ENGNaslov}
%\end{minipage}

Tema je bila že potrjena lani in je ponovno vložena: {\bf \textit{NE} }

Izjavljam, da je spodaj navedeni mentor predlog teme pregledal in odobril, ter da se z oddajo predloga strinja.

Magistrsko delo nameravam pisati v slovenščini. % In case you would like to write the thesis in English, comment this line out, and use the following template to explain your request:
%Komisijo zaprošam, da odobri pisanje magistrskega dela v angleškem jeziku z obrazložitvijo ... .

Za mentorja predlagam:

%%%%%%%%%%%%%%%%%%%%%%%%%% Izpolni kandidat! %%%%%%%%%%%%%%%%%%%%%%%%%%
\hfill\begin{minipage}{\dimexpr\textwidth-2cm}
Ime in priimek: doc. dr. Matevž Pesek\\
Ustanova: Fakulteta za računalništvo in informatiko, Univerza v Ljubljani\\
Elektronski naslov: matevz.pesek@fri.uni-lj.si
\end{minipage}
%%%%%%%%%%%%%%%%%%%%%%%%%% Konec izpolnjevanja %%%%%%%%%%%%%%%%%%%%%%%%%%

\bigskip


\hfill V Ljubljani, \today. \\
%V Ljubljani, dne …………………………
%
\hfill Podpis mentorja: \hspace{180px} \\ Podpis kandidata/kandidatke:    

\clearpage
\section*{PREDLOG TEME MAGISTRSKEGA DELA}

\section{Področje magistrskega dela}

slovensko: računalništvo in informatika, računalniška grafika, razširnjena resničnost, optično zaznavanje in sledenje predmetov, namizne igre\\
angleško: Computer Science and Informatics, Computer Graphics, Augmented Reality, Optical Object Detection and Tracking, Tabletop Games

\section{Ključne besede}

slovensko: obogatena resničnost, zaznava predmetov, sledenje predmetom, namizna igra \\
angleško: augmented reality, object detection, object tracking, tabletop game

\section{Opis teme magistrskega dela}

V službi imamo v skupnih prostorih namizno igro Osmica, ki jo za sprostitev, med delovnim časom lahko igramo
zaposleni in obiskovalci podjetja. Igramo po ameriških pravilih (in barvah krogel - 7 "polnih" in 7 "praznih" ter osmico), kjer pa si
pravila prilagajamo glede na sposobnost ter povprečnega igralnega časa, ki ponavadi igramo s trenutnimi igralci. Nemalokrat se prav tako
zgodi, da določenemu sodelavcu nenamerno zaprem pot do razpoložljivih krogel, ki jih lahko potisne v luknje. Posledično postane igra
nezanimiva, svoje pa doda še to, da nekih naprednih tehnik igranje ne poznamo, saj je to igranje v službi za večino edini stik z biljardom. 
Zato bi bilo zanimivo, da bi se igra izboljšala s pomočjo tehonologije, ki bi nam pomagala pri igranju tako, da bi spremljala situacijo na
mizim in nam vizualno pokazala, kaj je potrebno storiti, da bi bila igra bolj zanimiva in napeta. Vizualizirala bi se nam pot do 
krogle, ki jo trenutno lahko potisnemo v luknjo, prav tako pa bi nam pokazala, kje se bo krogla odbila, 
če bi jo potisnili v določeno smer in pod določenim kotom. \\Aplikcija v obogateni resničnosti, s pomočjo
očal za navidezno in razširjeno resničnost Meta Quest 3, bi nam pomagala za treniranje veščin za 
ameriška pravila osmice, predvsem na začetniškem nivoju igranja namizne igre. Gre se predvsem za implementacijo
mobilne rešitve za prepoznavo lokacij krogel, palice in poze dlani na resnični mizi in realnočasovni izris informacij
za zanimivejšo igro z manjhnimi podpornimi elementi ter primerjavo izdelanega sistema s trenutno dostopnimi sistemi,
ki delujejo v dvo-dimenzionalnem načinu.

% Navodilo (pobrišite v končnem izdelku):
% \cmnt{
% \textbf{Briši iz končnega izdelka:}
% Dolžina teme je zelo odvisna od zgoščenosti teksta in jasnosti podajanja argumentov. Zato je zelo težko predpisati natančno dolžino vsakega podpoglavja brez da bi ob tem posegali preveč v stil pisanja. Splošno vodilo naj bo, da naj bo iz teme: (i) jasno razviden problem in relevantnost problema, (ii) izpostavljene potencialne pomanjkljivosti sorodnih rešitev, (iii) novosti/prispevki naloge naj bodo jasni in v relaciji do sorodnih rešitev, (iv) jasne naj bodo pričakovane uporabljene metode za razvoj vaše rešitve, evalvacijo uspešnosti in primerjavo s sorodnimi deli.
% \\
% \\
% Kljub temu v vsakem podpoglavju podajamo okvirni obseg v številu besed. Vodilo pa naj bo vseeno vsebinska kvaliteta.}
% V nadaljevanju opredelite izhodišča magistrskega dela in utemeljite znanstveno ali strokovno relevantnost predlagane teme.

\newpage

\textbf{Pretekle potrditve predložene teme:}\\
Predložena tema ni bila oddana in potrjena v preteklih letih.

\newpage
\subsection{Uvod in opis problema}
\cmnt{Tole mi ni ravno jasno. A razširim samo na več besed?}
%Navodilo:
V službi imamo v skupnih prostorih namizno igro Osmica, ki jo za sprostitev, \textbf{med delovnim časom} lahko igramo
zaposleni in obiskovalci podjetja. Igramo po ameriških pravilih (in barvah krogel - 7 "polnih" in 7 "praznih" ter osmico), kjer pa si
pravila prilagajamo, glede na sposobnost ter povprečnega igralnega časa, ki ponavadi igramo s trenutnimi igralci. Ne malokrat se prav tako
zgodi, da določenemu sodelavcu nenamerno zaprem pot do razpoložljivih krogel, ki jih lahko potisne v luknje. Posledično postane igra
nezanimiva, svoje pa doda še to, da nekih naprednih tehnik igranje ne poznamo, saj je to igranje za večino edini stik z biljardom. 
Zato bi bilo zanimivo, da bi se igra izboljšala s pomočjo tehonologije, ki bi nam pomagala pri igranju tako, da bi spremljala situacijo na
mizi in nam vizualno pokazala, kaj je potrebno storiti, da bi bila igra bolj zanimiva in napeta. Vizualizirala bi se nam pot do 
krogle, ki jo trenutno lahko potisnemo v luknjo, prav tako pa bi nam pokazala, kje se bo krogla odbila, 
če bi jo potisnili v določeno smer in pod določenim kotom.  \\
Aplikcija v obogateni resničnosti bi nam pomagala za treniranje veščina za ameriška pravila osmice, predvsem na začetniškem nivoju igranja
namizne igre. 
% \cmnt{Pojasnite, kaj je problem, ki ga želite reševati, in podajte motivacijo za delo. Pri opisu motivacije se navežite na literaturo in nerešene probleme, ki jih bo naslavljala vaša magistrska naloga. Delo umestite v ožje področje dela. Okvirni obseg: ~300 besed (1/2 strani A4).}

\subsection{Pregled sorodnih del}

Preden sem načrtoval magistrsko delo, sem predlega obstoječa dela na področju obogatene resničnosti
za zadnjih 5 let, ki se osredotočajo na tak tip igre, pa tudi na samo področje zaznave in
spreljanja predmetov v 2D pogledu na mizo. WenKai et al. \cite{WenKai2024} se ukvarjajo s tehnologijami za
obdelavo barvnih fotografij mize za Osmico in algoritmi za umetno zaznavo krogel s pomočjo
globokega učenja. Predvsem me je zanimalo kako je mogoče zaznavati krogle, in kako se lahko določi
pozicija krogle. Slabost je seveda v tem, da je prisotno globoko učenje, kar za Meta
Quest 3 ni najbolj optimalno, saj bi bilo učenje v realnem času preveč zahtevno, če ne neizvedljivo.
Obrnil sem se na obstoječe reštive, ki deloma ponujajo že vgrajeno zaznavo objektov, kot je na primer
Microsoft Kinect V2 (v nadaljevanju Kinect V2), sploh zato, ker si omenjeno napravo lastim in poznam druge uporabe omenjene
kamere. Med drugim je to globinsko zaznavanje in sposobnost tro-dimenzionalnega zajemanja okolice. \\
Sousa et al. \cite{Sousa2016} so s pomočjo projektorja in kamere Kinect V2 razvili sistem,
ki krogle zazna, glede na kot pod katerim je palico pristavljena na kroglo, predvidi smer in kot odboja
na mizi in napovedi projicira na mizo. Sistem deluje v realnem času, ne glede na uporabljene palice,
krogle in material mize, hkrati pa so pogoji osvetlitve okolice zahtevni, saj niso kontrolirani, tako kot je
to pri profesionalnih turnirjih v biljardu (Ne vem kako delati "svoje komentarje": biljard ali osmicah - tu bi rabil
narediti nek prehod, ker bi raje uporabil biljard). Uspešnost napovedi je po ocenah avtorjev 98-odstotna, ni pa še bila preizkušena na višjih 
tekmovalnih nivojihigre, s čimer bi lahko dodatno evalviral in izboljšal sistem fizike (to moram še 1x preveriti) same igre.  \\
Kljub temu pa so pri omenjeni rešitvi pomanjkljivosti, saj sistem ne zna izračunati
sile s katero palica udari ob belo kroglo, kot tudi pod katero pozicijo jo udari, s čimer
dodatno izboljšamo natačnost napovedi, lahko pa s pomočjo tega izvedemo tehniko skakanja žogice, ki nam pride prav pri zapiranju nasprotnikov. \\
Čeprav je Kinect (ne glede) na različico strojne opreme dovršena tehnologija, pa jo je
podjetje Microsoft že od leta 2010, ko je splavilo prvo različico, obravnavalo mačehovsko,
saj za sistem Xbox 360 ni bilo dovolj iger, ki bi izkoriščale tehnologijo. Dodatna zahteva je bila
tudi, da je bila soba, kjer je bila naprava postavljena, dovolj velika. Podobne težave so se
se nadaljevale skozi naslednje generacije, kar je pripeljalo, do tega, da je Microsoft sistem
ukini, podizvajalec, ki je razvil sistem pa je bil kupljen s strani Apple-a. Ta tehnologija
predmetu Interaktivnost in oblikovanje informacij in je v obliki poročila: \url{https://unilj-my.sharepoint.com/:b:/g/personal/ak78348_student_uni-lj_si/Ec6HjsWnBCtKnxV41PBFEY4BLBBXPTgsY_bIx2kWkt_TqQ?e=0oGZRk}. \\

%Navodilo:
\cmnt{Opišite pregled sorodnih del na ožjem področju, na katerem nameravate opravljati magistrsko nalogo. Vsako delo naj bo na kratko opisano v nekaj stavkih, besedilo pa naj poudari njegove glavne prednosti, slabosti ali posebnosti. Sklicujte se na dela, navedena v razdelku \ref{literatura} Literatura in viri. Pregled naj bo fokusiran.  Okvirni obseg: 1/2 - 2/3 strani A4.}

\subsection{Predvideni prispevki magistrske naloge}

%Navodilo:
% \cmnt{Opišite predvidene prispevke magistrske naloge s področja računalništva in informatike, ki so lahko strokovni ali znanstveni. Poudarite in opišite predvideni napredek ali novost vašega dela v primerjavi z obstoječim stanjem na strokovnem (ali znanstvenem) področju.  Okvirni obseg: 70 besed.}
Predviden prispevek magistrske naloge s področja računalništva in informatike je uporaba
tehnologij obogatene resničnosti in naprednega zaznavanja predmetov in dlani za namene 
izobraževanja in zabave, hkrati pa posredno krepimo timski duh in socialne veščine 
zaposlenih. S pomočjo tehnologije, ki je dostopna in enostavna za uporabo ter ne zahteva
kompleksne postavitve sistema in njegovega vzdrževanja, se lahko tudi naredi nekaj na temo učinkovite rabe električne energije. Tako bi lahko sistem namesto s Kinectom
V2 priključenega na zmogljivejši računalnik,uporabljali le kamero telefona iPhone Pro Max, telefona katerega drugega proizvajalca ali
pa samo očala Meta Quest 3.
\subsection{Metodologija}

%Navodilo:
Za razvoj primarne aplikacije bom uporabil igralni pogon Unity 6, ki omogoča enostavno
razvijanje igra za različne platforme, med drugim tudi za očala Meta Quest 3 (v nadaljevanju samo Quest), ki tečejo na
operacijskem sistemu Android. Igralni pogon podpira razvoj v programskem jeziku C\# in
ogrodje .NET, razvojni okolji pa bosta Visual Studio in Visual Studio Code. Za zaznavanje krogel
se bo uporabljala kamera, ki se nahaja na Quest-u ali pa zunanja kamera, ki ima možnost
zaznavanja objektov, kot je na primer Microsoft Kinect V2. Slednji prav tako uporablja
tehnologijo zaznavanja globine, prav tako pa je možno uporabiti C\# in .NET za razvoj. Prenos in
obvedalava določenih podatkov bo odloženo na zunanji program ali spletno aplikacijo. Slednji bo beležila tudi
podatke o igri, ki bodo služili za izboljšanje algoritmov za napovedovanje gibanja krogel in samo držo dlani.
Pri nalogi bom stremel k čim manjšemu številu programskih in strojnh komponent, da bi bila aplikacija čim bolj preprosta za uporabo in razumevanje.
Čeprav bo večji del razvoja namanjen aplikaciji za Quest 3, bom preučil obstoječe možnosti, za prilagoditev aplikacije za 
Quest 2.
 \\
% \cmnt{Na kratko opredelite metodologijo, ki jo nameravate uporabiti pri svojem delu. Metodologija vsebuje metode, ki jih nameravate uporabiti (npr. razvoj v izbranem programskem jeziku, izdelava strojne opreme itd.), postopek analize, postopek evalvacije vašega prispevka in primerjavo s sorodnimi deli.  Okvirni obseg: 1/4 - 1/3 A4 strani.}

\subsection{Literatura in viri}
\label{literatura}

%Navodilo:
% \cmnt{Tu navedite vse vire, ki jih citirate v predlogu teme. Citiranje naj bo v skladu z znanstveno-strokovnimi standardi citiranja, na primer, \cite{Zivkovic2004}. Seznam naj vsebuje vsaj nekaj del, objavljenih v zadnjih petih letih. Prednostno naj bodo navedene objave s konferenc, revij, oziroma drugih priznanih virov.}

\renewcommand\refname{}
\vspace{-50px}
\bibliographystyle{elsarticle-num}
\bibliography{./bibliografija/bibliography}


%\bigskip
%
%Ljubljana, \today .

\end{document}
